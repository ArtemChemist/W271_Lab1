% Options for packages loaded elsewhere
\PassOptionsToPackage{unicode}{hyperref}
\PassOptionsToPackage{hyphens}{url}
%
\documentclass[
  11pt,
]{article}
\usepackage{amsmath,amssymb}
\usepackage{lmodern}
\usepackage{iftex}
\ifPDFTeX
  \usepackage[T1]{fontenc}
  \usepackage[utf8]{inputenc}
  \usepackage{textcomp} % provide euro and other symbols
\else % if luatex or xetex
  \usepackage{unicode-math}
  \defaultfontfeatures{Scale=MatchLowercase}
  \defaultfontfeatures[\rmfamily]{Ligatures=TeX,Scale=1}
\fi
% Use upquote if available, for straight quotes in verbatim environments
\IfFileExists{upquote.sty}{\usepackage{upquote}}{}
\IfFileExists{microtype.sty}{% use microtype if available
  \usepackage[]{microtype}
  \UseMicrotypeSet[protrusion]{basicmath} % disable protrusion for tt fonts
}{}
\makeatletter
\@ifundefined{KOMAClassName}{% if non-KOMA class
  \IfFileExists{parskip.sty}{%
    \usepackage{parskip}
  }{% else
    \setlength{\parindent}{0pt}
    \setlength{\parskip}{6pt plus 2pt minus 1pt}}
}{% if KOMA class
  \KOMAoptions{parskip=half}}
\makeatother
\usepackage{xcolor}
\usepackage[margin=1in]{geometry}
\usepackage{color}
\usepackage{fancyvrb}
\newcommand{\VerbBar}{|}
\newcommand{\VERB}{\Verb[commandchars=\\\{\}]}
\DefineVerbatimEnvironment{Highlighting}{Verbatim}{commandchars=\\\{\}}
% Add ',fontsize=\small' for more characters per line
\usepackage{framed}
\definecolor{shadecolor}{RGB}{248,248,248}
\newenvironment{Shaded}{\begin{snugshade}}{\end{snugshade}}
\newcommand{\AlertTok}[1]{\textcolor[rgb]{0.94,0.16,0.16}{#1}}
\newcommand{\AnnotationTok}[1]{\textcolor[rgb]{0.56,0.35,0.01}{\textbf{\textit{#1}}}}
\newcommand{\AttributeTok}[1]{\textcolor[rgb]{0.77,0.63,0.00}{#1}}
\newcommand{\BaseNTok}[1]{\textcolor[rgb]{0.00,0.00,0.81}{#1}}
\newcommand{\BuiltInTok}[1]{#1}
\newcommand{\CharTok}[1]{\textcolor[rgb]{0.31,0.60,0.02}{#1}}
\newcommand{\CommentTok}[1]{\textcolor[rgb]{0.56,0.35,0.01}{\textit{#1}}}
\newcommand{\CommentVarTok}[1]{\textcolor[rgb]{0.56,0.35,0.01}{\textbf{\textit{#1}}}}
\newcommand{\ConstantTok}[1]{\textcolor[rgb]{0.00,0.00,0.00}{#1}}
\newcommand{\ControlFlowTok}[1]{\textcolor[rgb]{0.13,0.29,0.53}{\textbf{#1}}}
\newcommand{\DataTypeTok}[1]{\textcolor[rgb]{0.13,0.29,0.53}{#1}}
\newcommand{\DecValTok}[1]{\textcolor[rgb]{0.00,0.00,0.81}{#1}}
\newcommand{\DocumentationTok}[1]{\textcolor[rgb]{0.56,0.35,0.01}{\textbf{\textit{#1}}}}
\newcommand{\ErrorTok}[1]{\textcolor[rgb]{0.64,0.00,0.00}{\textbf{#1}}}
\newcommand{\ExtensionTok}[1]{#1}
\newcommand{\FloatTok}[1]{\textcolor[rgb]{0.00,0.00,0.81}{#1}}
\newcommand{\FunctionTok}[1]{\textcolor[rgb]{0.00,0.00,0.00}{#1}}
\newcommand{\ImportTok}[1]{#1}
\newcommand{\InformationTok}[1]{\textcolor[rgb]{0.56,0.35,0.01}{\textbf{\textit{#1}}}}
\newcommand{\KeywordTok}[1]{\textcolor[rgb]{0.13,0.29,0.53}{\textbf{#1}}}
\newcommand{\NormalTok}[1]{#1}
\newcommand{\OperatorTok}[1]{\textcolor[rgb]{0.81,0.36,0.00}{\textbf{#1}}}
\newcommand{\OtherTok}[1]{\textcolor[rgb]{0.56,0.35,0.01}{#1}}
\newcommand{\PreprocessorTok}[1]{\textcolor[rgb]{0.56,0.35,0.01}{\textit{#1}}}
\newcommand{\RegionMarkerTok}[1]{#1}
\newcommand{\SpecialCharTok}[1]{\textcolor[rgb]{0.00,0.00,0.00}{#1}}
\newcommand{\SpecialStringTok}[1]{\textcolor[rgb]{0.31,0.60,0.02}{#1}}
\newcommand{\StringTok}[1]{\textcolor[rgb]{0.31,0.60,0.02}{#1}}
\newcommand{\VariableTok}[1]{\textcolor[rgb]{0.00,0.00,0.00}{#1}}
\newcommand{\VerbatimStringTok}[1]{\textcolor[rgb]{0.31,0.60,0.02}{#1}}
\newcommand{\WarningTok}[1]{\textcolor[rgb]{0.56,0.35,0.01}{\textbf{\textit{#1}}}}
\usepackage{graphicx}
\makeatletter
\def\maxwidth{\ifdim\Gin@nat@width>\linewidth\linewidth\else\Gin@nat@width\fi}
\def\maxheight{\ifdim\Gin@nat@height>\textheight\textheight\else\Gin@nat@height\fi}
\makeatother
% Scale images if necessary, so that they will not overflow the page
% margins by default, and it is still possible to overwrite the defaults
% using explicit options in \includegraphics[width, height, ...]{}
\setkeys{Gin}{width=\maxwidth,height=\maxheight,keepaspectratio}
% Set default figure placement to htbp
\makeatletter
\def\fps@figure{htbp}
\makeatother
\setlength{\emergencystretch}{3em} % prevent overfull lines
\providecommand{\tightlist}{%
  \setlength{\itemsep}{0pt}\setlength{\parskip}{0pt}}
\setcounter{secnumdepth}{5}
\ifLuaTeX
  \usepackage{selnolig}  % disable illegal ligatures
\fi
\IfFileExists{bookmark.sty}{\usepackage{bookmark}}{\usepackage{hyperref}}
\IfFileExists{xurl.sty}{\usepackage{xurl}}{} % add URL line breaks if available
\urlstyle{same} % disable monospaced font for URLs
\hypersetup{
  pdftitle={Lab 1, Short Questions},
  hidelinks,
  pdfcreator={LaTeX via pandoc}}

\title{Lab 1, Short Questions}
\author{}
\date{\vspace{-2.5em}}

\begin{document}
\maketitle

{
\setcounter{tocdepth}{2}
\tableofcontents
}
\begin{Shaded}
\begin{Highlighting}[]
\FunctionTok{library}\NormalTok{(tidyverse)}
\FunctionTok{library}\NormalTok{(patchwork)}
\FunctionTok{library}\NormalTok{(GGally)}
\FunctionTok{library}\NormalTok{(nnet)}
\FunctionTok{library}\NormalTok{(car)}
\end{Highlighting}
\end{Shaded}

\hypertarget{strategic-placement-of-products-in-grocery-stores-5-points}{%
\section{Strategic Placement of Products in Grocery Stores (5
points)}\label{strategic-placement-of-products-in-grocery-stores-5-points}}

These questions are taken from Question 12 of chapter 3 of the
textbook(Bilder and Loughin's ``Analysis of Categorical Data with R.

\begin{quote}
\emph{In order to maximize sales, items within grocery stores are
strategically placed to draw customer attention. This exercise examines
one type of item---breakfast cereal. Typically, in large grocery stores,
boxes of cereal are placed on sets of shelves located on one side of the
aisle. By placing particular boxes of cereals on specific shelves,
grocery stores may better attract customers to them. To investigate this
further, a random sample of size 10 was taken from each of four shelves
at a Dillons grocery store in Manhattan, KS. These data are given in the
}cereal\_dillons.csv \emph{file. The response variable is the shelf
number, which is numbered from bottom (1) to top (4), and the
explanatory variables are the sugar, fat, and sodium content of the
cereals.}
\end{quote}

\begin{Shaded}
\begin{Highlighting}[]
\NormalTok{cereal }\OtherTok{\textless{}{-}} \FunctionTok{read\_csv}\NormalTok{(}\StringTok{\textquotesingle{}../data/short{-}questions/cereal\_dillons.csv\textquotesingle{}}\NormalTok{)}
\NormalTok{cereal}
\end{Highlighting}
\end{Shaded}

\begin{verbatim}
## # A tibble: 40 x 7
##       ID Shelf Cereal                             size_g sugar_g fat_g sodium_mg
##    <dbl> <dbl> <chr>                               <dbl>   <dbl> <dbl>     <dbl>
##  1     1     1 Kellog's Razzle Dazzle Rice Crisp~     28      10   0         170
##  2     2     1 Post Toasties Corn Flakes              28       2   0         270
##  3     3     1 Kellogg's Corn Flakes                  28       2   0         300
##  4     4     1 Food Club Toasted Oats                 32       2   2         280
##  5     5     1 Frosted Cheerios                       30      13   1         210
##  6     6     1 Food Club Frosted Flakes               31      11   0         180
##  7     7     1 Capn Crunch                            27      12   1.5       200
##  8     8     1 Capn Crunch's Peanut Butter Crunch     27       9   2.5       200
##  9     9     1 Post Honeycomb                         29      11   0.5       220
## 10    10     1 Food Club Crispy Rice                  33       2   0         330
## # i 30 more rows
\end{verbatim}

\hypertarget{recode-data}{%
\subsection{Recode Data}\label{recode-data}}

(1 point) The explanatory variables need to be reformatted before
proceeding further (sample code is provided in the textbook). First,
divide each explanatory variable by its serving size to account for the
different serving sizes among the cereals. Second, rescale each variable
to be within 0 and 1. Construct side-by-side box plots with dot plots
overlaid for each of the explanatory variables. Also, construct a
parallel coordinates plot for the explanatory variables and the shelf
number. Discuss whether possible content differences exist among the
shelves.

\begin{Shaded}
\begin{Highlighting}[]
\NormalTok{stand01 }\OtherTok{\textless{}{-}} \ControlFlowTok{function}\NormalTok{(x)\{}
\NormalTok{  (x }\SpecialCharTok{{-}} \FunctionTok{min}\NormalTok{(x))}\SpecialCharTok{/}\NormalTok{(}\FunctionTok{max}\NormalTok{(x) }\SpecialCharTok{{-}} \FunctionTok{min}\NormalTok{(x))}
\NormalTok{\}}
\NormalTok{cereal2 }\OtherTok{\textless{}{-}} \FunctionTok{data.frame}\NormalTok{(}
  \AttributeTok{Shelf =}\NormalTok{ cereal}\SpecialCharTok{$}\NormalTok{Shelf,}
  \AttributeTok{Cereal =}\NormalTok{ cereal}\SpecialCharTok{$}\NormalTok{Cereal,}
  \AttributeTok{sugar =} \FunctionTok{stand01}\NormalTok{(}\AttributeTok{x =}\NormalTok{ cereal}\SpecialCharTok{$}\NormalTok{sugar\_g}\SpecialCharTok{/}\NormalTok{cereal}\SpecialCharTok{$}\NormalTok{size\_g),}
  \AttributeTok{fat =} \FunctionTok{stand01}\NormalTok{(}\AttributeTok{x =}\NormalTok{ cereal}\SpecialCharTok{$}\NormalTok{fat\_g}\SpecialCharTok{/}\NormalTok{cereal}\SpecialCharTok{$}\NormalTok{size\_g),}
  \AttributeTok{sodium =} \FunctionTok{stand01}\NormalTok{(}\AttributeTok{x =}\NormalTok{ cereal}\SpecialCharTok{$}\NormalTok{sodium\_mg}\SpecialCharTok{/}\NormalTok{cereal}\SpecialCharTok{$}\NormalTok{size\_g)}
\NormalTok{)}
\end{Highlighting}
\end{Shaded}

\begin{Shaded}
\begin{Highlighting}[]
\CommentTok{\# Sugar}
\FunctionTok{boxplot}\NormalTok{(}\AttributeTok{formula =}\NormalTok{ sugar }\SpecialCharTok{\textasciitilde{}}\NormalTok{ Shelf, }\AttributeTok{data =}\NormalTok{ cereal2, }\AttributeTok{ylab =} \StringTok{"Sugar"}\NormalTok{, }\AttributeTok{xlab =} \StringTok{"Shelf"}\NormalTok{, }\AttributeTok{pars =} \FunctionTok{list}\NormalTok{(}\AttributeTok{outpch =} \ConstantTok{NA}\NormalTok{))}
\FunctionTok{stripchart}\NormalTok{(}\AttributeTok{x =}\NormalTok{ cereal2}\SpecialCharTok{$}\NormalTok{sugar }\SpecialCharTok{\textasciitilde{}}\NormalTok{ cereal2}\SpecialCharTok{$}\NormalTok{Shelf, }\AttributeTok{lwd =} \DecValTok{2}\NormalTok{, }\AttributeTok{col =} \StringTok{"red"}\NormalTok{, }\AttributeTok{method =} \StringTok{"jitter"}\NormalTok{, }\AttributeTok{vertical =} \ConstantTok{TRUE}\NormalTok{, }\AttributeTok{pch =} \DecValTok{1}\NormalTok{, }\AttributeTok{add =} \ConstantTok{TRUE}\NormalTok{)}
\end{Highlighting}
\end{Shaded}

\includegraphics{short_questions_files/figure-latex/boxplots-1.pdf}

\begin{Shaded}
\begin{Highlighting}[]
\CommentTok{\# Fat}
\FunctionTok{boxplot}\NormalTok{(}\AttributeTok{formula =}\NormalTok{ fat }\SpecialCharTok{\textasciitilde{}}\NormalTok{ Shelf, }\AttributeTok{data =}\NormalTok{ cereal2, }\AttributeTok{ylab =} \StringTok{"Fat"}\NormalTok{, }\AttributeTok{xlab =} \StringTok{"Shelf"}\NormalTok{, }\AttributeTok{pars =} \FunctionTok{list}\NormalTok{(}\AttributeTok{outpch =} \ConstantTok{NA}\NormalTok{))}
\FunctionTok{stripchart}\NormalTok{(}\AttributeTok{x =}\NormalTok{ cereal2}\SpecialCharTok{$}\NormalTok{fat }\SpecialCharTok{\textasciitilde{}}\NormalTok{ cereal2}\SpecialCharTok{$}\NormalTok{Shelf, }\AttributeTok{lwd =} \DecValTok{2}\NormalTok{, }\AttributeTok{col =} \StringTok{"red"}\NormalTok{, }\AttributeTok{method =} \StringTok{"jitter"}\NormalTok{, }\AttributeTok{vertical =} \ConstantTok{TRUE}\NormalTok{, }\AttributeTok{pch =} \DecValTok{1}\NormalTok{, }\AttributeTok{add =} \ConstantTok{TRUE}\NormalTok{)}
\end{Highlighting}
\end{Shaded}

\includegraphics{short_questions_files/figure-latex/boxplots-2.pdf}

\begin{Shaded}
\begin{Highlighting}[]
\CommentTok{\# Sodium}
\FunctionTok{boxplot}\NormalTok{(}\AttributeTok{formula =}\NormalTok{ sodium }\SpecialCharTok{\textasciitilde{}}\NormalTok{ Shelf, }\AttributeTok{data =}\NormalTok{ cereal2, }\AttributeTok{ylab =} \StringTok{"Sodium"}\NormalTok{, }\AttributeTok{xlab =} \StringTok{"Shelf"}\NormalTok{, }\AttributeTok{pars =} \FunctionTok{list}\NormalTok{(}\AttributeTok{outpch =} \ConstantTok{NA}\NormalTok{))}
\FunctionTok{stripchart}\NormalTok{(}\AttributeTok{x =}\NormalTok{ cereal2}\SpecialCharTok{$}\NormalTok{sodium }\SpecialCharTok{\textasciitilde{}}\NormalTok{ cereal2}\SpecialCharTok{$}\NormalTok{Shelf, }\AttributeTok{lwd =} \DecValTok{2}\NormalTok{, }\AttributeTok{col =} \StringTok{"red"}\NormalTok{, }\AttributeTok{method =} \StringTok{"jitter"}\NormalTok{, }\AttributeTok{vertical =} \ConstantTok{TRUE}\NormalTok{, }\AttributeTok{pch =} \DecValTok{1}\NormalTok{, }\AttributeTok{add =} \ConstantTok{TRUE}\NormalTok{)}
\end{Highlighting}
\end{Shaded}

\includegraphics{short_questions_files/figure-latex/boxplots-3.pdf}

\begin{quote}
It appears that the items that are higher in sodium tend to be on shelf
\#1. Items that are high in sugar tend to go on shelf \#2. For fat,
there does not appear to be any obvious trends of which shelf items that
are high or low in fat go to.
\end{quote}

\begin{Shaded}
\begin{Highlighting}[]
\NormalTok{cereal2}\SpecialCharTok{$}\NormalTok{ShelfNumber }\OtherTok{\textless{}{-}} \FunctionTok{as.character}\NormalTok{(cereal2}\SpecialCharTok{$}\NormalTok{Shelf)}
\FunctionTok{ggparcoord}\NormalTok{(}\AttributeTok{data =}\NormalTok{ cereal2, }\AttributeTok{columns =} \DecValTok{3}\SpecialCharTok{:}\DecValTok{5}\NormalTok{, }\AttributeTok{groupColumn =} \StringTok{"ShelfNumber"}\NormalTok{, }\AttributeTok{showPoints =} \ConstantTok{TRUE}\NormalTok{, }\AttributeTok{scale =} \StringTok{"uniminmax"}\NormalTok{)}
\end{Highlighting}
\end{Shaded}

\includegraphics{short_questions_files/figure-latex/parallel coordinates plot-1.pdf}

\begin{quote}
The parallel coordinates plot is harder to glean information from than
the box plots. When looking closely one can find much of the same
information that was found in the box plots. Namely, the highest sugar
items belonging to shelf \#2 and the high sodium items being found in
shelf \#1.
\end{quote}

\begin{quote}
Without running an regression it is not responsible to say definitively
whether or not a content different exists between the shelves. Based on
these graphs there doesn't seem to be too big of a relationship between
sugar, fat, and sodium and the shelf a cereal belongs to.
\end{quote}

\hypertarget{evaluate-ordinal-vs.-categorical}{%
\subsection{Evaluate Ordinal
vs.~Categorical}\label{evaluate-ordinal-vs.-categorical}}

(1 point) The response has values of \(1, 2, 3,\) and \(4\). Explain
under what setting would it be desirable to take into account
ordinality, and whether you think that this setting occurs here. Then
estimate a suitable multinomial regression model with linear forms of
the sugar, fat, and sodium variables. Perform LRTs to examine the
importance of each explanatory variable. Show that there are no
significant interactions among the explanatory variables (including an
interaction among all three variables).

\begin{quote}
Ordinal data would make sense in the case where there's a natural
ordering to the shelves. Such as if higher shelves were inherently more
desirable than lower shelves. Then it would be expected that better
selling products be placed on higher shelves. It doesn't appear as if
this is the case here as the most desirable shelves would likely be at
eye level (towards the middle) and there doesn't appear to be any other
sort of ordering at play.
\end{quote}

\begin{Shaded}
\begin{Highlighting}[]
\NormalTok{model\_cereal\_shelves\_linear    }\OtherTok{\textless{}{-}} \FunctionTok{multinom}\NormalTok{(}\AttributeTok{formula =} \FunctionTok{factor}\NormalTok{(Shelf) }\SpecialCharTok{\textasciitilde{}}\NormalTok{ sugar }\SpecialCharTok{+}\NormalTok{ fat }\SpecialCharTok{+}\NormalTok{ sodium, }\AttributeTok{data =}\NormalTok{ cereal2, }\AttributeTok{family =} \StringTok{"multinomial"}\NormalTok{)}
\end{Highlighting}
\end{Shaded}

\begin{verbatim}
## # weights:  20 (12 variable)
## initial  value 55.451774 
## iter  10 value 37.329384
## iter  20 value 33.775257
## iter  30 value 33.608495
## iter  40 value 33.596631
## iter  50 value 33.595909
## iter  60 value 33.595564
## iter  70 value 33.595277
## iter  80 value 33.595147
## final  value 33.595139 
## converged
\end{verbatim}

\begin{Shaded}
\begin{Highlighting}[]
\NormalTok{model\_cereal\_shelves\_quadratic }\OtherTok{\textless{}{-}} \FunctionTok{multinom}\NormalTok{(}\AttributeTok{formula =} \FunctionTok{factor}\NormalTok{(Shelf) }\SpecialCharTok{\textasciitilde{}}\NormalTok{ sugar}\SpecialCharTok{*}\NormalTok{fat}\SpecialCharTok{*}\NormalTok{sodium, }\AttributeTok{data =}\NormalTok{ cereal2, }\AttributeTok{family =} \StringTok{"multinomial"}\NormalTok{)}
\end{Highlighting}
\end{Shaded}

\begin{verbatim}
## # weights:  36 (24 variable)
## initial  value 55.451774 
## iter  10 value 36.170336
## iter  20 value 31.166546
## iter  30 value 29.963705
## iter  40 value 28.414027
## iter  50 value 27.891712
## iter  60 value 27.763967
## iter  70 value 27.622579
## iter  80 value 27.438263
## iter  90 value 27.015534
## iter 100 value 26.772481
## final  value 26.772481 
## stopped after 100 iterations
\end{verbatim}

\begin{Shaded}
\begin{Highlighting}[]
\NormalTok{model\_cereal\_shelves\_linear}
\end{Highlighting}
\end{Shaded}

\begin{verbatim}
## Call:
## multinom(formula = factor(Shelf) ~ sugar + fat + sodium, data = cereal2, 
##     family = "multinomial")
## 
## Coefficients:
##   (Intercept)      sugar        fat    sodium
## 2    6.900708   2.693071  4.0647092 -17.49373
## 3   21.680680 -12.216442 -0.5571273 -24.97850
## 4   21.288343 -11.393710 -0.8701180 -24.67385
## 
## Residual Deviance: 67.19028 
## AIC: 91.19028
\end{verbatim}

\begin{Shaded}
\begin{Highlighting}[]
\NormalTok{model\_cereal\_shelves\_quadratic}
\end{Highlighting}
\end{Shaded}

\begin{verbatim}
## Call:
## multinom(formula = factor(Shelf) ~ sugar * fat * sodium, data = cereal2, 
##     family = "multinomial")
## 
## Coefficients:
##   (Intercept)      sugar       fat     sodium sugar:fat sugar:sodium fat:sodium
## 2   -4.563627   8.944868 22.063003   1.030077  35.60873   -12.250084  -23.75955
## 3   24.498320 -22.248456 35.981865 -27.899087 -17.12487    13.253103  -59.54150
## 4   27.246742 -21.852777  7.298799 -29.106797  41.08251     2.887805  -30.85250
##   sugar:fat:sodium
## 2        -55.88455
## 3         37.71571
## 4        -22.59552
## 
## Residual Deviance: 53.54496 
## AIC: 101.545
\end{verbatim}

\begin{Shaded}
\begin{Highlighting}[]
\NormalTok{lrt\_cereal\_main\_effects }\OtherTok{\textless{}{-}} \FunctionTok{Anova}\NormalTok{(model\_cereal\_shelves\_linear, }\AttributeTok{test =} \StringTok{"LR"}\NormalTok{)}
\NormalTok{lrt\_cereal\_main\_effects}
\end{Highlighting}
\end{Shaded}

\begin{verbatim}
## Analysis of Deviance Table (Type II tests)
## 
## Response: factor(Shelf)
##        LR Chisq Df Pr(>Chisq)    
## sugar   22.7648  3  4.521e-05 ***
## fat      5.2836  3     0.1522    
## sodium  26.6197  3  7.073e-06 ***
## ---
## Signif. codes:  0 '***' 0.001 '**' 0.01 '*' 0.05 '.' 0.1 ' ' 1
\end{verbatim}

\begin{Shaded}
\begin{Highlighting}[]
\NormalTok{lrt\_cereal\_quadratic\_effects }\OtherTok{\textless{}{-}} \FunctionTok{Anova}\NormalTok{(model\_cereal\_shelves\_quadratic, }\AttributeTok{test =} \StringTok{"LR"}\NormalTok{)}
\NormalTok{lrt\_cereal\_quadratic\_effects}
\end{Highlighting}
\end{Shaded}

\begin{verbatim}
## Analysis of Deviance Table (Type II tests)
## 
## Response: factor(Shelf)
##                  LR Chisq Df Pr(>Chisq)    
## sugar             19.2525  3  0.0002424 ***
## fat                6.1167  3  0.1060686    
## sodium            30.8407  3  9.183e-07 ***
## sugar:fat          3.2309  3  0.3573733    
## sugar:sodium       3.0185  3  0.3887844    
## fat:sodium         3.1586  3  0.3678151    
## sugar:fat:sodium   2.5884  3  0.4595299    
## ---
## Signif. codes:  0 '***' 0.001 '**' 0.01 '*' 0.05 '.' 0.1 ' ' 1
\end{verbatim}

\begin{quote}
Similar to what the graphs show there is clearly an obvious relationship
between sugar, sodium, and shelf placement. The Anova test shows that
none of the interaction terms are significant as well as the linear fat
variable.
\end{quote}

\hypertarget{where-do-you-think-apple-jacks-will-be-placed}{%
\subsection{Where do you think Apple Jacks will be
placed?}\label{where-do-you-think-apple-jacks-will-be-placed}}

(1 point) Kellogg's Apple Jacks (\url{http://www.applejacks.com}) is a
cereal marketed toward children. For a serving size of \(28\) grams, its
sugar content is \(12\) grams, fat content is \(0.5\) grams, and sodium
content is \(130\) milligrams. Estimate the shelf probabilities for
Apple Jacks.

\begin{Shaded}
\begin{Highlighting}[]
\NormalTok{stand02 }\OtherTok{\textless{}{-}} \ControlFlowTok{function}\NormalTok{(x, min, max)\{}
\NormalTok{  (x }\SpecialCharTok{{-}}\NormalTok{ min)}\SpecialCharTok{/}\NormalTok{(max }\SpecialCharTok{{-}}\NormalTok{ min)}
\NormalTok{\}}
\NormalTok{cereal3 }\OtherTok{\textless{}{-}} \FunctionTok{data.frame}\NormalTok{(}
  \AttributeTok{sugar =} \FunctionTok{stand02}\NormalTok{(}\DecValTok{12}\SpecialCharTok{/}\DecValTok{28}\NormalTok{, }\DecValTok{0}\NormalTok{, }\DecValTok{20}\NormalTok{),}
  \AttributeTok{fat =} \FunctionTok{stand02}\NormalTok{(}\FloatTok{0.5}\SpecialCharTok{/}\DecValTok{28}\NormalTok{, }\DecValTok{0}\NormalTok{, }\DecValTok{5}\NormalTok{),}
  \AttributeTok{sodium =} \FunctionTok{stand02}\NormalTok{(}\DecValTok{130}\SpecialCharTok{/}\DecValTok{28}\NormalTok{, }\DecValTok{0}\NormalTok{, }\DecValTok{330}\NormalTok{)}
\NormalTok{)}
\NormalTok{cereal3}
\end{Highlighting}
\end{Shaded}

\begin{verbatim}
##        sugar         fat     sodium
## 1 0.02142857 0.003571429 0.01406926
\end{verbatim}

\begin{Shaded}
\begin{Highlighting}[]
\NormalTok{pi.hat}\FloatTok{.1} \OtherTok{\textless{}{-}} \FunctionTok{predict}\NormalTok{(}\AttributeTok{object =}\NormalTok{ model\_cereal\_shelves\_linear, }\AttributeTok{newdata =}\NormalTok{ cereal3, }\AttributeTok{type =} \StringTok{"probs"}\NormalTok{)}

\NormalTok{aj\_shelf\_probs }\OtherTok{\textless{}{-}} \FunctionTok{round}\NormalTok{(}\FunctionTok{head}\NormalTok{(pi.hat}\FloatTok{.1}\NormalTok{), }\DecValTok{4}\NormalTok{)}
\NormalTok{aj\_shelf\_probs}
\end{Highlighting}
\end{Shaded}

\begin{verbatim}
##      1      2      3      4 
## 0.0000 0.0000 0.5918 0.4082
\end{verbatim}

\begin{quote}
The model predicts that Apple Jacks be placed on shelf \#3.
\end{quote}

\hypertarget{figure-3.3}{%
\subsection{Figure 3.3}\label{figure-3.3}}

(1 point) Construct a plot similar to Figure 3.3 where the estimated
probability for a shelf is on the \emph{y-axis} and the sugar content is
on the \emph{x-axis}. Use the mean overall fat and sodium content as the
corresponding variable values in the model. Interpret the plot with
respect to sugar content.

\begin{Shaded}
\begin{Highlighting}[]
\NormalTok{sugar }\OtherTok{\textless{}{-}} \FunctionTok{stand02}\NormalTok{(}\FunctionTok{seq}\NormalTok{(}\DecValTok{0}\NormalTok{, }\DecValTok{20}\NormalTok{, }\AttributeTok{by=}\FloatTok{0.1}\NormalTok{), }\DecValTok{0}\NormalTok{, }\DecValTok{20}\NormalTok{)}
\NormalTok{mean\_fat }\OtherTok{\textless{}{-}} \FunctionTok{replicate}\NormalTok{(}\DecValTok{201}\NormalTok{, }\FunctionTok{mean}\NormalTok{(}\FunctionTok{stand01}\NormalTok{(cereal2}\SpecialCharTok{$}\NormalTok{fat)))}
\NormalTok{mean\_sodium }\OtherTok{\textless{}{-}} \FunctionTok{replicate}\NormalTok{(}\DecValTok{201}\NormalTok{, }\FunctionTok{mean}\NormalTok{(}\FunctionTok{stand01}\NormalTok{(cereal2}\SpecialCharTok{$}\NormalTok{sodium)))}
\NormalTok{cereal4 }\OtherTok{\textless{}{-}} \FunctionTok{data.frame}\NormalTok{(}
  \AttributeTok{sugar =}\NormalTok{ sugar,}
  \AttributeTok{fat =}\NormalTok{ mean\_fat,}
  \AttributeTok{sodium =}\NormalTok{ mean\_sodium}
\NormalTok{)}
\NormalTok{pi.hat}\FloatTok{.2} \OtherTok{\textless{}{-}} \FunctionTok{predict}\NormalTok{(}\AttributeTok{object =}\NormalTok{ model\_cereal\_shelves\_linear, }\AttributeTok{newdata =}\NormalTok{ cereal4, }\AttributeTok{type =} \StringTok{"probs"}\NormalTok{)}

\NormalTok{plot.data }\OtherTok{\textless{}{-}} \FunctionTok{data.frame}\NormalTok{(}\AttributeTok{prediction =}\NormalTok{ pi.hat}\FloatTok{.2}\NormalTok{, }\AttributeTok{observed =} \FunctionTok{seq}\NormalTok{(}\DecValTok{0}\NormalTok{, }\DecValTok{20}\NormalTok{, }\AttributeTok{by=}\FloatTok{0.1}\NormalTok{))}
\NormalTok{shelf\_vs\_sugar\_plot }\OtherTok{\textless{}{-}} \FunctionTok{ggplot}\NormalTok{(plot.data, }\FunctionTok{aes}\NormalTok{(}\AttributeTok{x =}\NormalTok{ observed)) }\SpecialCharTok{+} 
  \FunctionTok{geom\_line}\NormalTok{(}\FunctionTok{aes}\NormalTok{(}\AttributeTok{y =}\NormalTok{ prediction}\FloatTok{.1}\NormalTok{)) }\SpecialCharTok{+}
  \FunctionTok{geom\_line}\NormalTok{(}\FunctionTok{aes}\NormalTok{(}\AttributeTok{y =}\NormalTok{ prediction}\FloatTok{.2}\NormalTok{), }\AttributeTok{linetype =} \StringTok{"dashed"}\NormalTok{) }\SpecialCharTok{+}
  \FunctionTok{geom\_line}\NormalTok{(}\FunctionTok{aes}\NormalTok{(}\AttributeTok{y =}\NormalTok{ prediction}\FloatTok{.3}\NormalTok{), }\AttributeTok{linetype =} \StringTok{"dotted"}\NormalTok{) }\SpecialCharTok{+}
  \FunctionTok{geom\_line}\NormalTok{(}\FunctionTok{aes}\NormalTok{(}\AttributeTok{y =}\NormalTok{ prediction}\FloatTok{.4}\NormalTok{), }\AttributeTok{linetype =} \StringTok{"twodash"}\NormalTok{) }\SpecialCharTok{+}
  \FunctionTok{labs}\NormalTok{(}\AttributeTok{title =} \StringTok{"Predicted Probabilities of Shelf"}\NormalTok{, }\AttributeTok{y =} \StringTok{"Probability"}\NormalTok{, }\AttributeTok{x =} \StringTok{"Sugar"}\NormalTok{)}

\NormalTok{shelf\_vs\_sugar\_plot}
\end{Highlighting}
\end{Shaded}

\includegraphics{short_questions_files/figure-latex/create figure 3.3-1.pdf}

\begin{quote}
As sugar increases the likelihood of being on shelf 1 or shelf 2
increases. When sugar is low the cereal is most likely to be placed on
shelf 3 or 4.
\end{quote}

\hypertarget{odds-ratios}{%
\subsection{Odds ratios}\label{odds-ratios}}

(1 point) Estimate odds ratios and calculate corresponding confidence
intervals for each explanatory variable. Relate your interpretations
back to the plots constructed for this exercise.

\begin{Shaded}
\begin{Highlighting}[]
\NormalTok{coefs}\FloatTok{.2} \OtherTok{\textless{}{-}} \FunctionTok{coef}\NormalTok{(model\_cereal\_shelves\_linear)[}\DecValTok{1}\NormalTok{,]}
\NormalTok{se}\FloatTok{.2} \OtherTok{\textless{}{-}} \FunctionTok{summary}\NormalTok{(model\_cereal\_shelves\_linear)}\SpecialCharTok{$}\NormalTok{standard.errors[}\DecValTok{1}\NormalTok{,]}
\NormalTok{ci}\FloatTok{.2} \OtherTok{\textless{}{-}} \FunctionTok{data.frame}\NormalTok{(}\AttributeTok{estimate =} \FunctionTok{exp}\NormalTok{(coefs}\FloatTok{.2}\NormalTok{), }\AttributeTok{lower =} \FunctionTok{exp}\NormalTok{(coefs}\FloatTok{.2} \SpecialCharTok{{-}} \FloatTok{1.96}\SpecialCharTok{*}\NormalTok{se}\FloatTok{.2}\NormalTok{), }\AttributeTok{upper =}  \FunctionTok{exp}\NormalTok{(coefs}\FloatTok{.2} \SpecialCharTok{+} \FloatTok{1.96}\SpecialCharTok{*}\NormalTok{se}\FloatTok{.2}\NormalTok{))}

\NormalTok{coefs}\FloatTok{.2} \OtherTok{\textless{}{-}} \FunctionTok{coef}\NormalTok{(model\_cereal\_shelves\_linear)[}\DecValTok{1}\NormalTok{,]}
\NormalTok{se}\FloatTok{.2} \OtherTok{\textless{}{-}} \FunctionTok{summary}\NormalTok{(model\_cereal\_shelves\_linear)}\SpecialCharTok{$}\NormalTok{standard.errors[}\DecValTok{1}\NormalTok{,]}
\NormalTok{ci}\FloatTok{.2} \OtherTok{\textless{}{-}} \FunctionTok{data.frame}\NormalTok{(}\AttributeTok{estimate =} \FunctionTok{exp}\NormalTok{(coefs}\FloatTok{.2}\NormalTok{), }\AttributeTok{lower =} \FunctionTok{exp}\NormalTok{(coefs}\FloatTok{.2} \SpecialCharTok{{-}} \FloatTok{1.96}\SpecialCharTok{*}\NormalTok{se}\FloatTok{.2}\NormalTok{), }\AttributeTok{upper =}  \FunctionTok{exp}\NormalTok{(coefs}\FloatTok{.2} \SpecialCharTok{+} \FloatTok{1.96}\SpecialCharTok{*}\NormalTok{se}\FloatTok{.2}\NormalTok{))}

\NormalTok{coefs}\FloatTok{.2} \OtherTok{\textless{}{-}} \FunctionTok{coef}\NormalTok{(model\_cereal\_shelves\_linear)[}\DecValTok{1}\NormalTok{,]}
\NormalTok{se}\FloatTok{.2} \OtherTok{\textless{}{-}} \FunctionTok{summary}\NormalTok{(model\_cereal\_shelves\_linear)}\SpecialCharTok{$}\NormalTok{standard.errors[}\DecValTok{1}\NormalTok{,]}
\NormalTok{ci}\FloatTok{.2} \OtherTok{\textless{}{-}} \FunctionTok{data.frame}\NormalTok{(}\AttributeTok{estimate =} \FunctionTok{exp}\NormalTok{(coefs}\FloatTok{.2}\NormalTok{), }\AttributeTok{lower =} \FunctionTok{exp}\NormalTok{(coefs}\FloatTok{.2} \SpecialCharTok{{-}} \FloatTok{1.96}\SpecialCharTok{*}\NormalTok{se}\FloatTok{.2}\NormalTok{), }\AttributeTok{upper =}  \FunctionTok{exp}\NormalTok{(coefs}\FloatTok{.2} \SpecialCharTok{+} \FloatTok{1.96}\SpecialCharTok{*}\NormalTok{se}\FloatTok{.2}\NormalTok{))}

\NormalTok{odds\_ratios }\OtherTok{\textless{}{-}} \StringTok{\textquotesingle{}fillin\textquotesingle{}}
\NormalTok{odds\_ratios}
\end{Highlighting}
\end{Shaded}

\begin{verbatim}
## [1] "fillin"
\end{verbatim}

\begin{quote}
`Fill this in: What do you learn about each of these variables?'
\end{quote}

\hypertarget{alcohol-self-esteem-and-negative-relationship-interactions-5-points}{%
\section{Alcohol, self-esteem and negative relationship interactions (5
points)}\label{alcohol-self-esteem-and-negative-relationship-interactions-5-points}}

Read the example \textbf{`Alcohol Consumption'} in chapter 4.2.2 of the
textbook(Bilder and Loughin's ``Analysis of Categorical Data with R).
This is based on a study in which moderate-to-heavy drinkers (defined as
at least 12 alcoholic drinks/week for women, 15 for men) were recruited
to keep a daily record of each drink that they consumed over a 30-day
study period. Participants also completed a variety of rating scales
covering daily events in their lives and items related to self-esteem.
The data are given in the \emph{DeHartSimplified.csv }data set.
Questions 24-26 of chapter 3 of the textbook also relate to this data
set and give definitions of its variables: the number of drinks consumed
(\texttt{numall}), positive romantic-relationship events
(\texttt{prel}), negative romantic-relationship events (\texttt{nrel}),
age (\texttt{age}), trait (long-term) self-esteem (\texttt{rosn}), state
(short-term) self-esteem (\texttt{state}).

The researchers stated the following hypothesis:

\begin{quote}
\emph{We hypothesized that negative interactions with romantic partners
would be associated with alcohol consumption (and an increased desire to
drink). We predicted that people with low trait self-esteem would drink
more on days they experienced more negative relationship interactions
compared with days during which they experienced fewer negative
relationship interactions. The relation between drinking and negative
relationship interactions should not be evident for individuals with
high trait self-esteem.}
\end{quote}

\begin{Shaded}
\begin{Highlighting}[]
\NormalTok{drinks }\OtherTok{\textless{}{-}} \FunctionTok{read\_csv}\NormalTok{(}\StringTok{\textquotesingle{}../data/short{-}questions/DeHartSimplified.csv\textquotesingle{}}\NormalTok{)}
\end{Highlighting}
\end{Shaded}

\hypertarget{eda}{%
\subsection{EDA}\label{eda}}

(2 points) Conduct a thorough EDA of the data set, giving special
attention to the relationships relevant to the researchers' hypotheses.
Address the reasons for limiting the study to observations from only one
day.

\begin{quote}
`Fill this in: What do you learn?'
\end{quote}

\hypertarget{hypothesis-one}{%
\subsection{Hypothesis One}\label{hypothesis-one}}

(2 points) The researchers hypothesize that negative interactions with
romantic partners would be associated with alcohol consumption and an
increased desire to drink. Using appropriate models, evaluate the
evidence that negative relationship interactions are associated with
higher alcohol consumption and an increased desire to drink.

\begin{quote}
`Fill this in: What do you learn?'
\end{quote}

\hypertarget{hypothesis-two}{%
\subsection{Hypothesis Two}\label{hypothesis-two}}

(1 point) The researchers hypothesize that the relation between drinking
and negative relationship interactions should not be evident for
individuals with high trait self-esteem. Conduct an analysis to address
this hypothesis.

\begin{quote}
`Fill this in: What do you learn?'
\end{quote}

\end{document}
